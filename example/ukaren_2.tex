%% ukaren_2.tex
%
% Copyright (C) 2023 conjikidow
%
% This work may be distributed and/or modified under the
% conditions of the LaTeX Project Public License, either version 1.3
% of this license or (at your option) any later version.
% The latest version of this license is in
%   https://www.latex-project.org/lppl.txt
% and version 1.3c or later is part of all distributions of LaTeX
% version 2008 or later.
%
% This work has the LPPL maintenance status `maintained'.
%
% The Current Maintainer of this work is conjikidow.
%
% This file is part of the "latex-templates-jsass" (The Work in LPPL)
% and all files in that bundle must be distributed together.


\documentclass{jsass-ukaren}

\usepackage{float}
\usepackage{graphicx}
\usepackage{siunitx}

\addbibresource{references/example.bib}


\begin{document}

\lectnum{1X01}
\title{第67回宇宙科学技術連合講演会原稿見本\\改行してもタイトル内で中央揃え}{Sample Format of Paper for the 67th Symposium on Space Science and Technology}

\author[1]{航空一郎}{Ichiro Koku}
\author[1]{宇宙花子}{Hanako Uchu}
\author[2]{航空次郎}{Jiro Koku}
\author[1]{航空三郎}{Saburo Koku}

\affiliation[1]{日本航空宇宙学会}{JSASS}
\affiliation[2]{東京大学}{The University of Tokyo}

\keywords{Space Science, Space Technology, Format Sample}

\begin{abstract}
  This paper instructs how to prepare your manuscript for the 67th Symposium on Space Sciences and Technology.
  All the final manuscripts should be written by word processors with the format specified in this manual.
  You are kindly requested to submit your manuscript in an electronic file on the JSASS website by \textcolor{red}{\textbf{10, August 2023}}.
\end{abstract}

\maketitle


\section{目的および背景}
  これは,第67回宇宙科学技術連合講演会講演集の原稿見本兼作成要領です.
  作成要領は,基本的に日本航空宇宙学会標準の講演会原稿作成要領によりますが,論文原稿の書式に統一性を持たせるために,今回ここに見本として,本作成要領を示します.
  両者に矛盾がある場合は,本稿が優先します.
  できる限り本稿と同じ,もしくは近い書式で作成していただきますようお願いいたします.

\section{引用の例}
  引用・文献にはBib\LaTeX を用いている。
  引用はこのように行える\cite{article_en_example,event_ja_example,online_ja_example}。


\printbibliography[title=参考文献]

\end{document}