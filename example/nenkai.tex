\documentclass{jsass-nenkai}


\begin{document}

\lectnum{1A01}

\title{日本航空宇宙学会講演集原稿書式見本}
\etitle{How to Prepare the Paper for the 49th Annual Meeting of JSASS}

\author{航空一郎,宇宙花子}
\institute{日本航空宇宙学会}
\eauthor{Ichiro Koku and Hanako Uchu}
\einstitute{JSASS}

\keywords{Society Activities, ...}

\begin{abstract}
  This is the manual for how to prepare your manuscript for the 49th Annual Meeting of Japan Society for Aeronautical and Space Science (JSASS).
  All the final drafts should be written by word processors with the format specified in this manua1.
  Your final paper in a PDF form must arrive at the Society Head-quarter by 16 February, 2018.
  Any questions regarding this manual should be addressed to the head-quarter.
\end{abstract}

\maketitle


\section{はじめに}
  これは航空宇宙学会主催の講演会講演集原稿の詳細および見本です。

\section{原稿の詳細}
  \subsection{原稿作成上の注意点}
    提出された原稿はそのまま写真製本用原稿として使用します。

    講演番号は、2月上旬にお知らせいたします。
    第50期年会講演会プログラムでご確認してください。

    \cref{tab:fotmat_of_manuscript}の原稿フォーマットにしたがって作成してください。
    図は印刷時に見えにくくならないよう、線は太目に作成してください。

    図表のタイトル記入場所は、図の場合は下端、表の場合は上端です。
    図・表には必ず番号とタイトルをつけてください。
    また、図の各軸には必ず変数名を記入してください。

    原稿にカラーの図を使用しても前刷集の印刷は白黒になります。

    \renewcommand{\arraystretch}{0.88}
    \begin{table}
      \centering
      \caption{原稿のフォーマット}
      \label{tab:fotmat_of_manuscript}
      \begin{tabular}{|p{57pt}|p{150pt}|} \hline
        原稿サイズ     & A4                                                           \\ \hline
        原稿ページ数   & 2〜10ページ                                                  \\ \hline
        原稿(PDF)      & 5MB以下                                                      \\
        サイズ         &                                                              \\ \hline
        原稿余白       & 左右各約23mm,2段組中央約7mm                                 \\
                       & 上下各約25mm                                                 \\ \hline
        日本語字体     & 明朝体                                                       \\ \hline
        文字サイズ     & 約9.5ポイント(9~10ポイント)                               \\ \hline
        行間           & 文字高の約50\%                                               \\ \hline
        段組           & 本文は2段組、タイトル,abstract部は段組なし                   \\ \hline
        題目           & 16ポイント明朝体、中央揃え                                   \\ \hline
        著者名         & 題名から2行送りで記入、中央揃え、登壇者氏名の左に〇印を付記  \\ \hline
        所属           & 氏名の後に括弧付きで記載                                     \\ \hline
        各段落の字下   & 1字下げ                                                     \\ \hline
        数字           & 英数字で半角                                                 \\ \hline
        英語字体       & Times new Roman                                              \\ \hline
        英文題目       & 日本語著者から2行送りで記載                                  \\ \hline
        英文著者名     & 英文題目に続いて記入する。中央揃え                           \\ \hline
        英語キーワード & 英文著者名から2行送り、Key Words : に続いて3~5語を記入      \\
                       & 中央揃え、ただし基準キーワード集から2~3語を選定する         \\ \hline
        英文概要       & 英文キーワードに続いて記載                                   \\ \hline
      \end{tabular}
      (この原稿は、9.5ポイントの明朝体で作成)
    \end{table}

\section{原稿送付}
  本会より、メール「講演会原稿アップロード依頼」でご案内いたしますアドレスへ、PDF原稿を電子投稿してください。\\
  原稿はA4判10ページ以内、PDF形式に変換、5MB以内。

  \vskip\baselineskip

  \textbf{※ PDF形式に変換した場合、フォント、画像の鮮明さ等、十分確認をしてから投稿してください。事務局では、文字化け等の責任は負いかねます。}

\section{原稿締切り}
  平成31年2月15日(金)

  \vskip\baselineskip
  \vskip\baselineskip

  \noindent *ご提出いただいた講演集掲載論文の著作権は、\\(一社)日本航空宇宙学会に帰属するものとします。
  ご了承ください。


\end{document}